
\usepackage[utf8]{inputenc}
% or whatever
% \usepackage{times}
\usepackage[T1]{fontenc}
% Or whatever. Note that the encoding and the font should match. If T1
% does not look nice, try deleting the line with the fontenc.

\usepackage{graphicx}      % include this line if your document contains figures
\usepackage{grffile}       % filenames
\usepackage{booktabs}       % filenames
\usepackage{amsmath}
\usepackage{mathtools}
\usepackage{amssymb}
\usepackage{xparse}
\usepackage{hyperref}
\usepackage{tikz}
\usepackage[english]{babel}
\usepackage{blindtext}
\usepackage{tikzscale}
\usepackage{bm}
\usepackage{ifthen}
\usepackage[ruled,noend]{algorithm2e}
\usepackage[justification=centering]{caption}
\usepackage{subcaption}
\usepackage[english]{babel}
\usepackage{pgfpages}

\usepackage{tikz}
\usetikzlibrary{calc,shapes,positioning}
\usetikzlibrary{overlay-beamer-styles}
\usetikzlibrary{plotmarks}
\usetikzlibrary{arrows.meta}
\usepackage{pgfplots}
\usepgfplotslibrary{patchplots}
\usepackage{amssymb}
\usepackage{booktabs}




\SetKwRepeat{Do}{do}{while}
\usepackage[author={Accacio}]{pdfcomment}

\usepackage[acronym]{glossaries}%

\newcommand{\acrSing}[3]{\newacronym{#1}{#2}{#3}
  \expandafter\newcommand\csname #1\endcsname{\gls{#1}}}

\newcommand{\acrPl}[5]{
  \newacronym[plural=#4,firstplural=#5 (#4)]{#1}{#2}{#3}
  \expandafter\newcommand\csname #1\endcsname{\gls{#1}}
  \expandafter\newcommand\csname #4\endcsname{\glspl{#1}}
}
\newcommand{\acr}[5][4=,5=]{
  \ifthenelse{\equal{#5}{}}
  {
    \acrSing{#1}{#2}{#3}
  }
  {
    \acrPl{#1}{#2}{#3}{#4}{#5}
  }
}

% \newcommand{\symbl}[3]{\newglossaryentry{#1}{name ={#2},	description ={#3}}
%   \expandafter\newcommand\csname #1\endcsname{\gls{#1}}
% }



\usepackage{color}

\usepackage{stfloats}

\newif\ifdebug%
\newcommand{\draft}{\debugtrue}
\newcommand{\final}{\debugfalse}
\newcommand{\question}[1]{\ifdebug%
  {\pdfcomment[color=red,opacity=0.4,subject=Should I?]{#1}} \fi}
\newcommand{\todo}[2][FORGOT TO DO SOMETHING]{\ifdebug%
  {\color{red}#2}\else \PackageError{}{#1}{#2}#2\fi}%
\newcommand\doing[2][FORGOT TO DO SOMETHING]{\ifdebug%
  {\color{blue}#2}\else \PackageError{}{#1}{#2}#2\fi}%
\newcommand\warning[1]{\ifdebug%
  {\color{red}#1}\fi}


%===============================================================================

\graphicspath{{../img/}}

\makeatletter
\hypersetup{
  bookmarks=true,         % show bookmarks bar?
  unicode=false,          % non-Latin characters in Acrobat’s bookmarks
  pdftoolbar=true,        % show Acrobat’s toolbar?
  pdfmenubar=true,        % show Acrobat’s menu?
  pdffitwindow=false,     % window fit to page when opened
  pdfstartview={FitH},    % fits the width of the page to the window
  pdftitle={\@title},    % title
  pdfauthor={\@author},     % author
  pdfsubject={},   % subject of the document
  pdfcreator={Rafael Accácio},   % creator of the document
  % pdfproducer={Producer}, % producer of the document
  pdfkeywords={keyword1, key2, key3}, % list of keywords
  pdfnewwindow=true,                  % links in new PDF window
  % colorlinks=true,            % false: boxed links; true: colored links
  % colorlinks=false,            % false: boxed links; true: colored links
  % linkcolor=black, % color of internal links (change box color with linkbordercolor)
  citecolor=black,          % color of links to bibliography
  filecolor=black,          % color of file links
  % urlcolor=black            % color of external links
}
\makeatother
\newtheorem{assumption}{Assumption}%[numberby]

\newif\ifwebcast
\webcastfalse

\newcommand{\webcast}{\webcasttrue}% Add circles to put webcam

% \newcommand<>{\script}[1]{\note#2{{#1}}}
\newcommand<>{\script}[1]{\note{\onslide#2{#1~\\}}}
